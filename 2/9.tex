\documentclass[a4paper,10pt]{book}
\usepackage[utf8]{inputenc}
\usepackage{amsmath}



\begin{document}

%Number of the question
\textbf{9.}

%Intro
\textbf{Intro} \par
In this question we have to exercise our capacity of getting some information about
conditional independence and see if we can infer something more based on it.\par

%%%%%%%%%%%%%%%%%%%%%%%%%%%%%%%%%%%%%%%%%%%%%%%%%%%%%%%%%%%%%%%%%%%%%%%%%%%%%%%%%%%%%%

%Solution
\textbf{Solution} \par

a)

We have two information about conditional independence. The first one is:

\[X \perp W | Z, Y\]

Based on this, we can state that:

%Equation 1
\begin{equation}
\begin{split}
p(x, w | z, y) = p(x| z, y)p(w| z, y)
\end{split}
\end{equation}

The second one is:

\[X \perp Y | Z\]

Based on this, we can state that:

%Equation 2
\begin{equation}
\begin{split}
p(x, y | z) = p(x|z)p(y|z)
\end{split}
\end{equation}

Now, we have to see if $X \perp Y, W | Z$ is true.

%Equation 3
\begin{equation}
\begin{split}
p(x,y,w|z) = p(x|z)p(y|x,z)p(w|x,y,z) = \\
p(x|z)p(y|z)p(w|y,z) = p(x|z)p(y,w|z)
\end{split}
\end{equation}

In (3), the first passage is the chain rule of probability. The second 
make use of the conditional independences given to us (Y do not depend on X given Z and W do not 
depend on X given Z and Y). The third passage is putting the joint distribution back together.
Therefore, the proposition is \textbf{true}.\par


b)

Again, we have two information about conditional independence. The first one is:

\[X \perp Y | Z\]

Based on this, we can state that:

%Equation 4
\begin{equation}
\begin{split}
p(x, y | z) = p(x|z)p(y|z)
\end{split}
\end{equation}

The second one is:

\[X \perp Y | W\]

Based on this, we can state that:

%Equation 5
\begin{equation}
\begin{split}
p(x, y | w) = p(x|w)p(y|w)
\end{split}
\end{equation}

Now, we have to see if $X \perp Y | Z, W$ is true.

%Equation 6
\begin{equation}
\begin{split}
p(x,y|z,w) = p(x|z,w)p(y|z,w)
\end{split}
\end{equation}

In (6), we used the fact that X and Y are conditionally independent given Z or W.
So, the proposition is \textbf{true}.\par






%Conclusion
\textbf{Conclusion} \par
There is not a major conclusion to be mentioned. As it was said it the begin, I think the main
purpose of this exercise is just to exercise the brain in inference involving conditional independence.

\end{document}


